\RequirePackage[latest]{latexrelease}
\documentclass[a4paper,12pt]{article}
\usepackage[utf8]{inputenc}
\usepackage[T1]{fontenc}
\usepackage[a4paper]{geometry}
\usepackage{amsmath}
\usepackage{amssymb}
\usepackage{graphicx}
\usepackage{rotating}
\usepackage{placeins}
\usepackage[slovak]{babel}
\usepackage{makeidx}
\usepackage[colorlinks=true,linkcolor=blue,urlcolor=black]{hyperref}
\usepackage{bookmark}
\usepackage{bm}
\usepackage{cleveref}
%\usepackage{tikz}
\usepackage{multicol}
\usepackage{wrapfig}
%\usepackage{booktab}
\usepackage{array}
\usepackage{siunitx}
\usepackage{lmodern}
\usepackage{ellipsis}
\usepackage{nicefrac}
%\usepackage{microtype}
%\usetikzlibrary{positioning}

\crefname{equation}{rovn.}{rovn.}
\Crefname{equation}{Rovn.}{Rovn.}
\crefname{figure}{obr.}{obr.}
\Crefname{figure}{Obr.}{Obr.}
\crefname{table}{tab.}{tab.}
\Crefname{table}{Tab.}{Tab.}
\newcommand{\crefrangeconjunction}{ - }
\renewcommand{\figurename}{Obr.}
\renewcommand{\tablename}{Tab.}
\newcommand{\overbar}[1]{\mkern 1.5mu\overline{\mkern-1.5mu#1\mkern-1.5mu}\mkern 1.5mu}
\newcommand{\overhat}[1]{\mkern 1.5mu\hat{\mkern-1.5mu#1\mkern-1.5mu}\mkern 1.5mu}
\newcommand{\abs}[1]{|#1|}
\newcommand{\lint}{\int\limits}
\newcommand{\lsum}{\sum\limits}


\begin{document}
    % \section{Princíp spätnovezbovej ilnearizácie}
    % Pred vysvetlením princípu spätnovezobnej linearizácie vstupno výstupnej, uľahčuje našu prácu vysvetliť, princíp spätnovazbovej linearizácie ako takej.

    % Zoberme si systém daný stavový opisom \cref{eqn:svlvv_sys0}.
    % \begin{equation}
    %     \begin{gathered}
    %     \begin{pmatrix} 
    %         \dot{x}_1 \\ 
    %         \dot{x}_2
    %     \end{pmatrix} = \begin{pmatrix}
    %         2x_2 + x_2^2 + u \\
    %         x_2 + 3x_1
    %     \end{pmatrix}\\
    %     y = x_1
    %     \end{gathered}
    %     \label{eqn:svlvv_sys0}
    % \end{equation}
    % Všimnime si, že tento systém je nelineárny, kvôli členu $x_2^2$ nachádzajúceho s v rovnici pre $\dot{x}_1$. Avšak zavedením pravidla $u = v - x_2^2$, kde $v$ je nový vstup do systému (my sme si ho zaviedli), dostaneme nový systém daný \cref{eqn:svlvv_sys1}.
    % \begin{equation}
    %     \begin{gathered}
    %     \begin{pmatrix} 
    %         \dot{x}_1 \\ 
    %         \dot{x}_2
    %     \end{pmatrix} = \begin{pmatrix}
    %         2x_2 + v \\
    %         x_2 + 3x_1
    %     \end{pmatrix}\\
    %     y = x_1
    %     \end{gathered}
    %     \label{eqn:svlvv_sys1}
    % \end{equation}
    % Tento systém narozdiel od pôvodného, je lineárny, čo nám dovoľuje pri návrhu regulátora využiť metódy vyvynuté pre lineárne systémy.

    \section{Princíp spätnovezbovej linearizácie \\ vstupno výstupnej}
    Predpokladajme systém so stavovým opisom z \cref{eqn:svlvv_sys2}.
    \begin{equation}
        \begin{gathered}
        \begin{pmatrix} 
            \dot{x}_1 \\ 
            \dot{x}_2
        \end{pmatrix} = \begin{pmatrix}
            x_1^2 + u \\
            x_2^3 + x_1
        \end{pmatrix}\\
            y = x_1
        \end{gathered}
        \label{eqn:svlvv_sys2}
    \end{equation}
    Naším cieľom je nájsť pravidlo, podľa ktorého ak budeme meniť vstup do systému, teda veličinu $u$, tak výstup zo systému bude sledovať predpísanú hodnotu $y_d$, čo je vo všeobecnosti funkciou času $t$(zapisujeme $y_d(t)$).

    V ďalšom kroku chceme odvodiť závislosť $y = f(u, \overbar{x})$, lebo potom vieme zaviesť pravidlo $u = f^{-1}(y_d(t), \overbar{x})$ (inverzná závislosť) z čoho máme $y = y_d(t)$, čo je naším cieľom.

    Ak systém je $n$-tého rádu, tak bude platiť, že maximálne n-tá derivácia $y$ je priamo závislá od $u$ kedže ak by to tak nebolo, tak by to bol systém vyššieho rádu. Máme teda spôsob ako nájsť závislosť $ y = f(u) $, a to deriváciou vzťahu pre $y$, teda $y = x_1$ v príklade vyššie, pričom derivovať budeme musieť maximálne toľko krát, koľkého rádu je systém, teda v príklade vyššie je to maximálne 2 krát. Prevedme teraz tieto výpočty.

\end{document}

