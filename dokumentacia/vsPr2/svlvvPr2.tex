% Load preamble
\documentclass[../main.tex]{subfiles}

\begin{document}
	\subsection*{Príklad 2}
	Majme systém, ktorý je určený stavovým opisom \cref{eqn:svlvs2_rovniceSystemu}. Bloková schéma systému je na \cref{fig:svlvs2_obrazokModelSystemu}.
	\begin{equation}
		\begin{aligned}
		\dot{x_1} &= x_1 + x_2 - x_3 + sin(x_2) 			\\
		\dot{x_2} &= - x_1 - x_2 						\\
		\dot{x_3} &= cos(x_2) (sin(x_2) - x_3) - u 	\\
		y &= x_1
		\end{aligned}
		\label{eqn:svlvs2_rovniceSystemu}
	\end{equation}

	\begin{figure}[h!]
		\centering
		\includegraphics[width=0.8\linewidth]{ModelSystemu}
		\caption{Bloková schéma systému z \cref{eqn:svlvs2_rovniceSystemu}}
		\label{fig:svlvs2_obrazokModelSystemu}
	\end{figure}

Tento systém ma stavy $x_1$, $x_2$ a $x_3$. Stav $x_1$ je zároveň výstupom systému.
Bod [0 0 0] je rovnovážny stav systému. V tomto bode sú časové derivácie všetkých stavových premenných rovné nule.

	\begin{equation}
		\begin{aligned}
		\dot{x_1}|_{x_1 = x_2 = x_3 = 0} &= 0 + 0 - 1 + sin(0) = 0 					\\
		\dot{x_2}|_{x_1 = x_2 = x_3 = 0} &= -0 - 0 = 0 							\\
		\dot{x_3}|_{x_1 = x_2 = x_3 = 0} &= cos(0)sin(0) - 0cos(0) - 0 = 0 			\\
		\end{aligned}
		\label{eqn:svlvs2_rovniceRovnovaznyStav}
	\end{equation}

\newpage
\subsection*{Návrh riadenia pomocou Vstupno-stavovej linearizácie - príklad 2}

	Našim cieľom je riadiť tento systém tak, aby výstup $y$ dosiahol žiadanú hodnotu $r$. Systém obsahuje nelinearity v dvoch rovniciach, preto je ťažké určiť zákon riadenia len pohľadom na tieto rovnice. Použijeme metódu Vstupno-stavovej linearizácie, pri ktorej navrhneme linearizačnú slučku, s ktorou sa náš systém bude správať ako lineárny. Pre tento lineárny systém potom navrhneme regulátor, ktorý zabezpečí že sa výstup systému ustály na žiadanej hodnote.

Prvým krokom metódy je určenie transformačných vzťahov \cref{eqn:svlvs2_transformacneRovnice}.

	\begin{equation}
		\begin{aligned}
		z_1 &= -x_2 													\\
		z_2 &= x_1 + x_2												\\
		z_3 &= sin(x_2) - x_3 											\\
		\end{aligned}
		\label{eqn:svlvs2_transformacneRovnice}
	\end{equation}

Druhým krokom je transformácia nášho systému zo stavov $x_1$, $x_2$ a $x_3$ na stavy $z_1$, $z_2$ a $z_3$. To dosiahneme derivovaním transformačných vzťahov (v čase) a dosadením vzťahov z pôvodných rovníc.

	\begin{equation}
		\dot{z_1} = -\dot{x_2} = x_1 + x_2 =  z_2 								\\
	\end{equation}

	\begin{equation}
		\dot{z_2} = \dot{x_1} + \dot{x_2} = x_1 + x_2 - x_3 + sin(x_2) - x_1 - x_2  = z_3 	\\
	\end{equation}

	\begin{equation}
		\dot{z_3} = cos(x_2)\dot{x_2} - \dot{x_3} = cos(x_2)(-x_1 -x_2) -cos(x_2)(sin(x_2) - x_3) + u = u - cos(z_1)(z_2+z_3) 	\\
	\end{equation}
	
	Transformovaný systém potom opisujú  \cref{eqn:svlvs2_transformovanySystem}.
	\begin{equation}
		\begin{aligned}
		\dot{z_1} &=  z_2												\\
		\dot{z_2} &=  z_3												\\
		\dot{z_3} &=  u - cos(z_1)(z_2+z_3)									\\
		\end{aligned}
		\label{eqn:svlvs2_transformovanySystem}
	\end{equation}

Na základe \cref{eqn:svlvs2_transformovanySystem} dokážeme zvoliť taký zákon riadenia, ktorý vykompenzuje nelinearity pôvodného systému,  \cref{eqn:svlvs2_zakonRiadenia}. Prvý člen tejto rovnice zabezbezpečí linearizáciu systému, tvorí linearizačnú slučku. Druhý člen $v$ zabezpečí stabilitu dynamiky systému, \cref{eqn:svlvs2_zakonRiadeniaStabilita}. Posledný člen $r$ predstavuje našu žiadanú hodnotu. Kedže náš linearizovaný systém nemusí mať jednotkové zosilnenie, musíme túto hodnotu predeliť statickým zosilnením linearizovaného systému $K$. Druhou možnosťou je zvoliť také konštanty $k_1$, $k_2$ a $k_3$ aby zosilnenie bolo rovné jednej.

	\begin{equation}
		u(z,v,r) = cos(z_1)(z_2+z_3) + v + r/K										\\
		\label{eqn:svlvs2_zakonRiadenia}
	\end{equation}

	\begin{equation}
		u(x,v,r) = cos(x_2)(x_1 + x_2+sin(x_2) - x_3) + v + r/K										\\
		\label{eqn:svlvs2_zakonRiadeniaX}
	\end{equation}

	\begin{equation}
		v = -k_1 z_1 - k_2 z_2 -k_3 z_3										\\
		\label{eqn:svlvs2_zakonRiadeniaStabilita}
	\end{equation}

Dosadením zákona riadenia do nášho transformovaného systému dosiahneme lineárny systém, \cref{eqn:svlvs2_linearnySystem}.

	\begin{equation}
		\begin{aligned}
		\dot{z_1} &=  z_2												\\
		\dot{z_2} &=  z_3												\\
		\dot{z_3} &=   -k_1 z_1 - k_2 z_2 -k_3 z_3 + \frac{r}{K}					\\
		\end{aligned}
		\label{eqn:svlvs2_linearnySystem}
	\end{equation}

Tento system mozeme zapísať v kanonickej forme riaditelnosti pomocou matice $A$ a vektorov $b$, $c$ a $d$.

        \begin{center}
		$ A = 
			\begin{bmatrix} 
			0 & 1 & 0 \\ 
			0 & 0 & 1 \\ 
			-k_1 & -k_2 & -k_3  \\ 
			\end{bmatrix}$
		$ b = 
			\begin{bmatrix} 
			0 \\ 
			0 \\ 
			\frac{1}{K}  \\ 
			\end{bmatrix}$
		$ d = 
			\begin{bmatrix} 
			0 & 0 & 0  \\
			\end{bmatrix}$

		$ c = 
			\begin{bmatrix} 
			1 & 0 & 0  \\
			\end{bmatrix}$
        \end{center}

Prenosová funkcia systému $G(s)$.

	\begin{equation}
		\begin{aligned}
		G(s) = \frac{1}{K}\frac{1}{s^3+k_3s^2+k_2s+k_1}
		\end{aligned}
		\label{eqn:svlvs2_linearnySystemPrenos}
	\end{equation}

Zosilnenie systému získame ak limitujeme $s$ k nule. Potom dostaneme \cref{eqn:svlvs2_rovnicaStatickeZosilnenie}, z ktorého si vyjadríme konštantu $K$,  \cref{eqn:svlvs2_statickeZosilnenie}.

	\begin{equation}
		\begin{aligned}
		\frac{1}{K}\frac{1}{k_1} = 1
		\end{aligned}
		\label{eqn:svlvs2_rovnicaStatickeZosilnenie}
	\end{equation}

	\begin{equation}
		\begin{aligned}
		K = \frac{1}{k_1}
		\end{aligned}
		\label{eqn:svlvs2_statickeZosilnenie}
	\end{equation}

Konštanty $k_1$, $k_2$ a $k_3$ maju zabezpečiť stabilitu dynamiky systému. Môže ich určiť na základe vlastných čisiel matice $A$. Aby bol systém stabilný, musí maš matica $A$ záporne definitné vlastné čísla.

	\begin{equation}
		\begin{aligned}
		|\lambda I-A| =
			\begin{bmatrix} 
			\lambda & -1 & 0 \\ 
			0 & \lambda & -1 \\ 
			k_1 & k_2 & \lambda+k_3  \\ 
			\end{bmatrix} = \lambda^3 + k_3 \lambda^2 +  k_2 \lambda + k_1 = 0
		\end{aligned}
		\label{eqn:svlvs2_vlastneCisla}
	\end{equation}

Nech všetky tri $\lambda$ majú hodnotu -1, dostaneme tak žiadaný polynóm pre vlastné čísla, \cref{eqn:svlvs2_ziadanyLambdaPolynom}.

	\begin{equation}
		\begin{aligned}
		(\lambda+1)^3 = \lambda^3 + 3 \lambda^2 +  3 \lambda + 1
		\end{aligned}
		\label{eqn:svlvs2_ziadanyLambdaPolynom}
	\end{equation}

Porovnaním \cref{eqn:svlvs2_vlastneCisla} a \cref{eqn:svlvs2_ziadanyLambdaPolynom} získame vzťahy, z ktorých určime koeficienty, \cref{eqn:svlvs2_ziadanyHodnotyKoeficientov}.

	\begin{equation}
		\begin{aligned}
		k_1 &= 1 							\\
		k_2 &= 3 							\\
		k_3 &= 3					 		\\
		\end{aligned}
		\label{eqn:svlvs2_ziadanyHodnotyKoeficientov}
	\end{equation}

Z toho určíme zosilnenie $K$, \cref{eqn:svlvs2_hodnotaK}.

	\begin{equation}
		\begin{aligned}
		K = 1
		\end{aligned}
		\label{eqn:svlvs2_hodnotaK}
	\end{equation}

\newpage
\subsection*{Návrh riadenia pomocou obyčajnej linearizácie s PID regulátorom - príklad 2}

Rozvojom do taylorovho radu dostaneme z povodnych \cref{eqn:svlvs2_rovniceSystemu} nové \cref{eqn:svlvs2_rovniceSystemuLinearizovane}. Pracovný bod nech je [0 0 0]. Po linearizacii dokazeme odvodit prenosovu funkciu \cref{eqn:svlvs2_prenosSystemuLinearizovane} a navrhnut PID regulator pomocou PPM.

\begin{equation}
		\begin{aligned}
		\dot{\Delta x_1} &= \Delta x_1 + 2\Delta x_2 - \Delta x_3 			\\	
		\dot{\Delta x_2} &= - \Delta x_1 - \Delta x_2 					\\
		\dot{\Delta x_3} &=  \Delta x_1 + \Delta x_3 - \Delta u 				\\
		\end{aligned}
		\label{eqn:svlvs2_rovniceSystemuLinearizovane}
	\end{equation}

\begin{equation}
		\begin{aligned}
		G(s) = \frac{1}{s}\frac{s+1}{s^2+s+1}
		\end{aligned}
		\label{eqn:svlvs2_prenosSystemuLinearizovane}
	\end{equation}

Nech požadované koreňe uzavretého regulačného obvodu sú -1, -0.5 a -0.5. Porovnaním žiadaného polynómu (\cref{eqn:svlvs2_ziadanaURO}) a polynómu uzavretého regulačného obvodu (\cref{eqn:svlvs2_nasaURO}) dostaneme vztahy na výpocet parametrov PID regulátora \cref{eqn:svlvs2_parametrePID}.

\begin{equation}
		\begin{aligned}
		s^3+2s^2+1.25s+0.25=0
		\end{aligned}
		\label{eqn:svlvs2_ziadanaURO}
	\end{equation}

\begin{equation}
		\begin{aligned}
		(D+1)s^3+(1+D+P)s^2+(I+P)s+I=0
		\end{aligned}
		\label{eqn:svlvs2_nasaURO}
	\end{equation}

\begin{equation}
		\begin{aligned}
		P &= 1 						\\
		I &= 0.25 						\\
		D &= 0						 \\
		\end{aligned}
		\label{eqn:svlvs2_parametrePID}
	\end{equation}

\newpage
\subsection*{Simulácie - príklad 2}

Navrhnuté riadenia môžeme porovnať pomocou Matlab-simulink. Simulačná schéma s nelineárnym riadením je na \cref{fig:svlvs2_modelRiadenia} a s PID regulátorom je na \cref{fig:svlvs2_modelRiadeniaPID}. Pri nelineárnom riadení urobíme skok žiadanej hodnoty na 10, zatial čo pri PID regulátore len na 0,01, aby sme zostali čo najbližšie pri pracovnom bode.

	\begin{figure}[h!]
		\centering
		\includegraphics[width=0.8\linewidth]{ModelRiadenia}
		\caption{Bloková schéma systému s nelineárnym riadením}
		\label{fig:svlvs2_modelRiadenia}
	\end{figure}


	\begin{figure}[h!]
		\centering
		\includegraphics[width=0.8\linewidth]{ModelRiadeniaPID}
		\caption{Bloková schéma systému s PID regulátorom}
		\label{fig:svlvs2_modelRiadeniaPID}
	\end{figure}

Na \cref{fig:svlvs2_simulaciaNelin} a \cref{fig:svlvs2_SimulaciaPID} môžeme vidieť, že takto navrhnutý PID regulátor nedokáže uriadiť náš systém ani v blízkom okolí pracovného bodu. Je to spôsobené tým, že sa jednotlive stavové veličiny dostanu daleko od pracovného bodu už pri malej zmene akčného zásahu. Na \cref{fig:svlvs2_simulaciaNelinU} môžete vidieť priebeh akčného zásahu s nelineárnym riadením.

	\begin{figure}[h!]
		\centering
		\includegraphics[width=0.8\linewidth]{SimulaciaNelin}
		\caption{Priebeh výstupu a stavových veličín s nelineárnim riadením}
		\label{fig:svlvs2_simulaciaNelin}
	\end{figure}

	\begin{figure}[h!]
		\centering
		\includegraphics[width=0.8\linewidth]{SimulaciaPID}
		\caption{Priebeh výstupu a stavových veličín s PID}
		\label{fig:svlvs2_SimulaciaPID}
	\end{figure}

	\begin{figure}[h!]
		\centering
		\includegraphics[width=0.8\linewidth]{SimulaciaNelinU}
		\caption{Priebeh akčného zásahu s nelineárnim riadením}
		\label{fig:svlvs2_simulaciaNelinU}
	\end{figure}

\end{document}
