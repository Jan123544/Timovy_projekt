\newpage
\ifthenelse {\boolean{bachelor}}
{
	%\section{Conslusions}
	\section{Záver}
    V tímovom projekte sme sa zaoberali vytvorením niekoľkých cvičných modelov systémov, na ktorých sme potom ukázali vybrané metódy syntézy nelineárneho riadenia.

    Tak ako sme vyššie demonštrovali, pre riadenie niektorých systémov môže byť využitie nelineárneho riadenia nevyhnutné, v takýchto prípadoch nám metóda vstupno-stavovej linearizácie a metóda vstupno-výstupnej linearizácie poskytujú spôsob ako riadenie syntetizovať, v iných prípadoch však riadenie pomocou lineárneho PID regulátora, ktorý má parametre navrhnuté pre systém linearizovaný v rovnovažnom stave, postačuje a je jednoduchší z hľadiska návrhu. Preto je nutné pre každý systém určiť, či je riadenie pomocou PID regulátora postačujúce a ak nie, využiť napríklad jednu z vyššie uvedených metód.

    V prípade použitia uvedených metód návrhu nelineárneho riadenia, je potrebné analyzovať aj výsledný tvar nelineárneho zákona riadenia, pretože tento nemusí byť definovaný, pre všetky hodnoty stavových premenných.

    Počas projektu sme museli riešiť aj rôzne problémy spojené s prácou v tíme. Napríklad určenie času stretnutí vyhovujúci všetkým členom tímu, zvoliť spôsob diaľkovej komunikácie, organizovanie obsahu stretnutí a ďalšie. Vzhľadom na situáciu, ktorá sa vyskytla, sme boli nútený organizovať stretnutia tímu cez internet. 

}
%{
	%\chapter{Conclusions}
%	\chapter{Záver}
%}

