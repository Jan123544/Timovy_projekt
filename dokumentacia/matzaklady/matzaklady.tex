% Load preamble
\documentclass[../main.tex]{subfiles}

\begin{document}

\subsection{Stavový opis}
	
\subsection{Rovnvoažné stavy}
	
\subsection{Linearizácia}
	
\subsection{Riešenie preurčenej sústavy rovníc}

Preurčená sústava rovníc, obsahuje viac rovníc ako neznámych premenných. Pri riešení preurčene sústavy rovníc použijeme metódu najmenších štvorcov. Pomocou metódy najmenších štvorcov sa budeme snažiť aproximovať riešenie $x$ preurčenej sústavy rovníc \ref{eqn:MaticovyZapisPSR}.

\begin{equation}
	Ax \approx b
	\label{eqn:MaticovyZapisPSR}
\end{equation}

Uvažujeme nasledovnú preurčenú sústavu rovníc:

\begin{equation}
	\begin{split}
	 x+y  & = 5 \\
	 2x+4y+10 & = 8 \\
	 x+5y & = 15 \\
	 -2x+4y+10 & = 8 \\
	\end{split}
	\label{eqn:PreurcenySystem}
\end{equation}

Preurčenú sústavu rovníc \ref{eqn:PreurcenySystem} môžeme maticovo zapísať v tvare:
\begin{equation}
\begin{bmatrix} 1 & 1\\ 2 & 4 \\1 &5 \\-2& 4\end{bmatrix}\begin{bmatrix}x \\y \end{bmatrix} = \begin{bmatrix} 5 \\-2\\15\\-2 \end{bmatrix}
 \label{eqn:MaticovyZapisPiklad}
\end{equation}

Na výpočet neznámych premenných x,y použijeme metódu najmenších štvorcov :
\begin{equation}
	\begin{split}
	\begin{bmatrix}x \\y \end{bmatrix} &=  (A^TA)^{-1}A^Tb\\
	\begin{bmatrix}x \\y \end{bmatrix} &=  \begin{bmatrix} 1.4265 \\0.9559\end{bmatrix}
	\end{split}
	\label{eqn:MNS}
\end{equation}

		
\end{document}
