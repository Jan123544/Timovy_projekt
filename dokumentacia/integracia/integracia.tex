\documentclass[../main.tex]{subfiles}
\begin{document}
    Vyššie uvedené príklady, môžu byť integrované do predmetu Riadenie Nelineárnych systémov niekoľkými spôsobmi. Jednou z možností je prezentovanie týchto príkladov priamo na prednáške.
    
    Ďalším priestorm pre aplikáciu týchto príkladov sú cvičenia. Kde by študenti mohli počas cvičení precvičovať metódy návrhu regulátorov pre tieto príklady.

    Alebo na začiatok, môžu byť poskytnuté ako dodatočný materiál s vypracovanými príkladmi k prednáškam alebo cvičeniam ako návod, na precvičenie si metód návrhu. A na základe odozvy študentov na tieto príklady by postupne mohli byť integrované do predchádzajúcich dvoch učebných procesov (prednášky, cvičenia). 

    Keďže obsahujú aj porovnanie s PID, čo zahŕňa návrh lin. regulátora v rovnovažnom stave linearizáciou, môžu byť využité na porovnanie lin. riadenia s nelineárnym, ale  aj ako príklady alebo návody na precvičenie a odkontorlovanie linearizácie. 

    Ďalším produktom tímového projektu je animácia riadenia polohy kyvadla v programe Matlab. Túto animáciu je možné použit ako pedagogickú pomôcku, ktorá by mohla pomôcť študentom so získaním predstavy o prejednávanej problematike a možných riešeniach predmetu RNS.

    %K dokumentácií je priložená, okrem prezentácie projektu aj prezentácia príkladov s riešeniami, ktorú je možné využiť na prednáške priamo.

\end{document}
