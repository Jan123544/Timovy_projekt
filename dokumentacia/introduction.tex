\newpage
\ifthenelse {\boolean{bachelor}}
{
	%\section{Introduction}
	\section{Úvod}
}
{
	%\chapter{Introduction}
%	\chapter{Úvod}
}

    V tímovom projekte sa venujeme definovaniu niekoľkých modelov systémov v stavovom opise, vhodných na demonštráciu návrhu nelineárneho riadenia pomocou metód vstupno-stavovej a vstupno-výstupnej linearizácie a návrhom nelineárne riadenia pre dané modely. Navrhnuté nelineárne riadenie porovnávame s lineárnym PID regulátorom a uvádzame aj základné matematické princípy, využívané pri návrhu pomocou uvedených metód.

    V časti \ref{sec:matematicke_zaklady} sa venujeme niektorým matematickým princípom, ktorých znalosť je nevyhnutná pre pochopenie metód, ale aj ďalšieho textu, v celom rozsahu. Konkrétne táto časť zahŕňa opakovanie k nasledujúcim témam: stavový opis systému, rovnovažné stavy a linearizácia. 
    
    Časť \ref{sec:vsl} sa zaoberá návrhom nelineárneho riadenia pomocou metódy vstupno-stavovej linearizácie. Sú tu prezentované dva príklady, ku každému je vypracovaný návrh riadenia danou metódou a pre porovnanie je navrhnutý aj PID regulátor pre linearizáciu systému v rovnovažnom stave.

    Návrhu pomocou metódy vstupno-výstupnej linearizácie sa venujeme v časti \ref{sec:vvl}. Podobne ako v predchádzajúcej časti, aj tu sa venujeme okrem návrhu pomocou hlavnej metódy aj návrhu PID regulátora pre linearizovaný systém.

    V časti \ref{sec:integracia} stručne opíšeme spôsoby ako by sa daný materiál dal integrovať do predmetu Riadenie Nelineárnych Systémov.

    V časti \ref{sec:riadproj} sa venujeme predstaveniu riešiteľského kolektívu, plánu projektu, dohodnutým metódam práce a záznamom o stretnutí. 

    Príkladáme aj program písaný v matlabe, spolu s manuálom na použitie, vhodný na ukážku navrhnutých zákonov riadenia pre jednoduchý príklad riadenia polohy matematického kyvadla. V programe sú implementované nelineárne zákony riadenia a aj dva PID regulátory navrhnuté pre linearizáciu kyvadla v stabilnom a v nestabilnom rovnovažnom stave. 


