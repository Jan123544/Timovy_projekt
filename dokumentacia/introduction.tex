\newpage
\ifthenelse {\boolean{bachelor}}
{
	%\section{Introduction}
	\section{Úvod}
}
{
	%\chapter{Introduction}
%	\chapter{Úvod}
}

    V tímovom projekte sa venujeme definovaniu neikoľkých modelov systémov v stavovom opise, vhodných na demonštráciu návrhu nelineárneho riadenia pomocou metód vstupno-stavovej a vstupno-výstupnej linearizácie. Navrhnuté nelineárne riadenie porovnávame s lineárnym PID regulátorom a uvádzame aj základné matematické princípy, využívané pri návrhu pomocou uvedených metód.

    V časti \ref{sec:matzaklady} sa venujeme niektorým matematickým princípom, ktorých znalosť je nevyhnutná pre pochopenie metód, ale aj ďalšieho textu, v celom rozsahu. Konkrétne táto časť zahrňa opakovanie k nasledujúcim témam: stavový opis systému, rovnovažné stavy a linearizácia. 
    
    Časť \ref{sec:vsl} sa zaoberá návrhom nelineárneho riadenia pomocou metódy vstupno-stavovej linearizácie. Sú tu prezentované dva príklady, ku každému je vypracovaný návrh riadenia danou metódou a riadenie systému PID navrhnuté pre linearizáciu v rovnovažnom stave.

    Návrhu pomocou metódy vstupno-výstupnej linearizácie sa venujeme v časti \ref{sec:vvl}. Podobne ako v predchádzajúcej časti, aj tu sa venujeme okrem návrhu pomcou hlavnej metódy aj návrhu PID regulátora pre linearizovaný systém.

    V časti \ref{sec:anim} opisujeme program, písaný v matlabe, vhodný na kvalitatívne porovnanie navrhnutých riadení na jednoduchom prípade riadenia polohy matematického kyvadla. V programe sú implementované nelineárne formy raidenia a aj PID navrhnuté pre linearizáciu kyvadla v stabilon aj nestabilnom rovnovažnom stave.

    Opisu spôsobu, akým by sa mohol, v tomto dokumente, prezentovaný materiál integrovať do predmetu Riadenie Nelineárnych Systémov, sa venujeme v časti \ref{sec:integracia}.
