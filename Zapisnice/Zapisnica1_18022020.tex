% Load preamble
\RequirePackage[latest]{latexrelease}
\documentclass[a4paper,12pt]{article}
\usepackage[utf8]{inputenc}
\usepackage[T1]{fontenc}
\usepackage[a4paper]{geometry}
\usepackage{amsmath}
\usepackage{amssymb}
\usepackage{graphicx}
\usepackage{rotating}
\usepackage{placeins}
\usepackage[slovak]{babel}
\usepackage{makeidx}
\usepackage[colorlinks=true,linkcolor=blue,urlcolor=black]{hyperref}
\usepackage{bookmark}
\usepackage{bm}
\usepackage{cleveref}
%\usepackage{tikz}
\usepackage{multicol}
\usepackage{wrapfig}
%\usepackage{booktab}
\usepackage{array}
\usepackage{siunitx}
\usepackage{lmodern}
\usepackage{ellipsis}
\usepackage{nicefrac}
%\usepackage{microtype}
%\usetikzlibrary{positioning}

\crefname{equation}{rovn.}{rovn.}
\Crefname{equation}{Rovn.}{Rovn.}
\crefname{figure}{obr.}{obr.}
\Crefname{figure}{Obr.}{Obr.}
\crefname{table}{tab.}{tab.}
\Crefname{table}{Tab.}{Tab.}
\newcommand{\crefrangeconjunction}{ - }
\renewcommand{\figurename}{Obr.}
\renewcommand{\tablename}{Tab.}
\newcommand{\overbar}[1]{\mkern 1.5mu\overline{\mkern-1.5mu#1\mkern-1.5mu}\mkern 1.5mu}
\newcommand{\overhat}[1]{\mkern 1.5mu\hat{\mkern-1.5mu#1\mkern-1.5mu}\mkern 1.5mu}
\newcommand{\abs}[1]{|#1|}
\newcommand{\lint}{\int\limits}
\newcommand{\lsum}{\sum\limits}

\usepackage{parskip}% http://ctan.org/pkg/parskip
\begin{document}
	\textbf{{\Huge Zápisnica č.1}}\\
			
	\textbf{Dátum:} 18.2.2020\\	
		
	\textbf{Účastníci:} prof. Ing. Ján Murgaš, PhD., Bc. Eva Štalmachová, Bc. Ján Urdianyk, Bc. Denis Vasko, Bc. Marek Trebuľa, členovia druhého tímu\\
		
	\textbf{Miesto:} D332, budova FEI STU\\	
	
	\textbf{Čas:} 10:30    
    \section*{Obash stretnutia}
    \begin{enumerate}
    	\item Prof. Ján Murgaš predstavil tímom predmet "Tímový projekt" a oboznámil nás s ideou tohto predmetu.
    	\item Na stretnutí sme si na vyzvanie prof. Jána Murgaša zvolili za kapitána tímu Bc. Denisa Vaska. Ako kapitán tímu je zodpovedný za komunikáciu s profesorom Murgašom a následnú distribúciu informácií medzi ostatných členov tímu.
    	\item Prof. Ján Murgaš nám zadal úlohu do ďalšieho stretnutia.
    	\item Prof. Ján Murgaš prisľúbil dodanie zadaní pre tímy na nasledujúcom\\ stretnutí.
    	\item Určili sme termín ďalšieho stretnutia s prof. Jánom Murgašom.
    	\item Tímovo sme sa dohodli, že podrobnejšie rozdelenie úloh medzi členov tímu preberieme po spresnení zadania prof. Murgašom.
    	\item Bc. Ján Urdianyk vytvoril git repozitár, kde budeme zálohovať, ukladať všetky súbory a informácie súvisiace s tímovým projektom.
    	\item  Bc. Denis Vasko vytvoril facebookovú skupinu pre náš tím. Určená je na rýchlu komunikáciu a distribúciu informácií ohľadom tímového projektu.
    \end{enumerate}    
    \section*{Úlohy}
    \begin{itemize}
    	\item Individuálne si naštudovať informácie o predmete "Tímový projekt" na stránke ÚRK.
    \end{itemize}

    \section*{Nasledujúce stretnutie}
    Naplánované na 25. februára 2020 o 10:00 v miestnosti D332 aj s prof. Jánom Murgašom
    
    \noindent\rule{15cm}{0.4pt}
   {\small 	\textbf{Vypracoval:} Bc. Marek Trebuľa\\
   \textbf{Dňa:} 22.2.2020 }
    

\end{document}
