% Load preamble
\RequirePackage[latest]{latexrelease}
\documentclass[a4paper,12pt]{article}
\usepackage[utf8]{inputenc}
\usepackage[T1]{fontenc}
\usepackage[a4paper]{geometry}
\usepackage{amsmath}
\usepackage{amssymb}
\usepackage{graphicx}
\usepackage{rotating}
\usepackage{placeins}
\usepackage[slovak]{babel}
\usepackage{makeidx}
\usepackage[colorlinks=true,linkcolor=blue,urlcolor=black]{hyperref}
\usepackage{bookmark}
\usepackage{bm}
\usepackage{cleveref}
%\usepackage{tikz}
\usepackage{multicol}
\usepackage{wrapfig}
%\usepackage{booktab}
\usepackage{array}
\usepackage{siunitx}
\usepackage{lmodern}
\usepackage{ellipsis}
\usepackage{nicefrac}
%\usepackage{microtype}
%\usetikzlibrary{positioning}

\crefname{equation}{rovn.}{rovn.}
\Crefname{equation}{Rovn.}{Rovn.}
\crefname{figure}{obr.}{obr.}
\Crefname{figure}{Obr.}{Obr.}
\crefname{table}{tab.}{tab.}
\Crefname{table}{Tab.}{Tab.}
\newcommand{\crefrangeconjunction}{ - }
\renewcommand{\figurename}{Obr.}
\renewcommand{\tablename}{Tab.}
\newcommand{\overbar}[1]{\mkern 1.5mu\overline{\mkern-1.5mu#1\mkern-1.5mu}\mkern 1.5mu}
\newcommand{\overhat}[1]{\mkern 1.5mu\hat{\mkern-1.5mu#1\mkern-1.5mu}\mkern 1.5mu}
\newcommand{\abs}[1]{|#1|}
\newcommand{\lint}{\int\limits}
\newcommand{\lsum}{\sum\limits}

\usepackage{parskip}% http://ctan.org/pkg/parskip
\usepackage[dvipsnames]{xcolor}
\begin{document}
	\textbf{{\Huge Zápisnica č.7}}\\
			
	\textbf{Dátum:} 30.3.2020\\	
		
	\textbf{Účastníci:} Bc. Eva Štalmachová, Bc. Ján Urdianyk, Bc. Denis Vasko, Bc. Marek Trebuľa

	\textbf{Miesto:} Online, Microsoft Teams\\	
	
	\textbf{Čas:} 18:00 (trvanie cca 40 minút)
	
	\textbf{\textcolor{red}{Z dôvodu pandémie ochorenia COVID-19 sa toto stretunie uskutočnilo online, keďže karanténne opatrenia to neumožnili inak.}}   
    \section*{Obsah stretnutia}
    \begin{enumerate}
    	\item Do stretnutia, sa nepodarilo spojiť s prof. Murgašom. 
    	\item Bc. Eva Štalmachová doprogramovala  do GUI reguláciu metódou vstupno-stavovej spätnoväzbovej linearizácie.
    	\item Bc. Ján Urdianyk dokončil matematickú časť.
    	\item Bc. Marek Trebuľa pripravil osnovu používateľskej dokumentácie.
    	\item Bc. Denis Vasko prešiel ešte raz všetky príklady a pripravil ich na prípadnú kontrolu prof. Murgašom. Navrhol doplniť prametre k1, k2 do GUI pri metóde vstupno-výstupnej spätnoväzbovej linearizácie. 
    	\item Rozdelili sa úlohy medzi členov tímu.
    	\item Dohodlo sa ďalšie stretnutie v rámci tímu, ktoré opäť prebehne kvôli pandémii pravdepodobne online.
    \end{enumerate}    
    \section*{Úlohy}
    \begin{itemize}
    	\item Bc. Denis Vasko má za úlohu kontaktovať prof. Jána Murgaša a dohodnúť ďalší postup.
    	\item Bc. Eva Štalmachová pripraví finálnu šablónu dokumentu v LaTex-u.
    	\item Bc. Ján Urdianyk sa pokúsi optimalizovať GUI.
    	\item Bc. Marek Trebuľa doprogramuje do GUI ovládače pre parametre k1, k2 pre metódu vstupno-výstupnej spätnoväzbovej linearizácie . 
    \end{itemize}

    \section*{Nasledujúce stretnutie}
    
    Stretnutie tímu online 6.4. s vypracovanými podúlohami v MS Teams.

    
    \noindent\rule{15cm}{0.4pt}
   {\small 	\textbf{Vypracoval:} Bc. Marek Trebuľa\\
   \textbf{Dňa:} 1.4.2020 }
    

\end{document}