% Load preamble
\RequirePackage[latest]{latexrelease}
\documentclass[a4paper,12pt]{article}
\usepackage[utf8]{inputenc}
\usepackage[T1]{fontenc}
\usepackage[a4paper]{geometry}
\usepackage{amsmath}
\usepackage{amssymb}
\usepackage{graphicx}
\usepackage{rotating}
\usepackage{placeins}
\usepackage[slovak]{babel}
\usepackage{makeidx}
\usepackage[colorlinks=true,linkcolor=blue,urlcolor=black]{hyperref}
\usepackage{bookmark}
\usepackage{bm}
\usepackage{cleveref}
%\usepackage{tikz}
\usepackage{multicol}
\usepackage{wrapfig}
%\usepackage{booktab}
\usepackage{array}
\usepackage{siunitx}
\usepackage{lmodern}
\usepackage{ellipsis}
\usepackage{nicefrac}
%\usepackage{microtype}
%\usetikzlibrary{positioning}

\crefname{equation}{rovn.}{rovn.}
\Crefname{equation}{Rovn.}{Rovn.}
\crefname{figure}{obr.}{obr.}
\Crefname{figure}{Obr.}{Obr.}
\crefname{table}{tab.}{tab.}
\Crefname{table}{Tab.}{Tab.}
\newcommand{\crefrangeconjunction}{ - }
\renewcommand{\figurename}{Obr.}
\renewcommand{\tablename}{Tab.}
\newcommand{\overbar}[1]{\mkern 1.5mu\overline{\mkern-1.5mu#1\mkern-1.5mu}\mkern 1.5mu}
\newcommand{\overhat}[1]{\mkern 1.5mu\hat{\mkern-1.5mu#1\mkern-1.5mu}\mkern 1.5mu}
\newcommand{\abs}[1]{|#1|}
\newcommand{\lint}{\int\limits}
\newcommand{\lsum}{\sum\limits}

\usepackage{parskip}% http://ctan.org/pkg/parskip
\begin{document}
	\textbf{{\Huge Zápisnica č.2}}\\
			
	\textbf{Dátum:} 25.2.2020\\	
		
	\textbf{Účastníci:} prof. Ing. Ján Murgaš, PhD., Bc. Eva Štalmachová, Bc. Ján Urdianyk, Bc. Denis Vasko, Bc. Marek Trebuľa, členovia druhého tímu\\
		
	\textbf{Miesto:} D332, budova FEI STU\\	
	
	\textbf{Čas:} 10:00    
    \section*{Obash stretnutia}
    \begin{enumerate}
    	\item Prof. Ján Murgaš oboznámil jednotlivé tímy s ich zadaniami.
    	\item Prof. Ján Murgaš poslal zadania e-mailom kapitánom tímov a odovzdal\\ tímom aj papierovú verziu.
    	\item Prof. Ján Murgaš nám zadal úlohu do ďalšieho stretnutia.
    	\item Po diskusií v tíme sme určili, že Bc. Marek Trebuľa bude spisovať zápisnice z jednotlivých stretnutí.
    	\item Určili sme termín ďalšieho stretnutia s prof. Jánom Murgašom.
    	\item Dohodli sme sa, že do stretnutia s prof. Jánom Murgašom prebehne\\ minimálne jedno stretnutie samostatne, kde splníme úlohu, ktorú sme dostali.
    	\item Bc. Eva Štalmachová prebrala papierové zadanie a rozdistribuovala ho medzi ostatných členov tímu, prostredníctvom facebookovej skupiny.
    	
    \end{enumerate}    
    \section*{Úlohy}
    \begin{itemize}
    	\item Do nasledujúceho stretnutia s prof. Jánom Murgašom prepracovať plán\\ s konkrétnymi úlohami a návrhmi riešení, ktoré budú rozdelené medzi\\ jednotlivých členov tímu. Tento plán úloh sa bude konzultovať s prof. Jánom Murgašom na nasledujúcom stretnutí.
    \end{itemize}

    \section*{Nasledujúce stretnutie}
    Individuálne stretnutie členov nášho tímu pred stretnutím s prof. Jánom\\ Murgašom. Miesto a čas si bližšie určíme v našom internetovom komunikačnom kanáli.
    
    Naplánované na 3. marca 2020 o 10:00 v miestnosti D332 aj s prof. Jánom Murgašom.
    
    \noindent\rule{15cm}{0.4pt}
   {\small 	\textbf{Vypracoval:} Bc. Marek Trebuľa\\
   \textbf{Dňa:} 26.2.2020 }
    

\end{document}
