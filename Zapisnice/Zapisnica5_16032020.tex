% Load preamble
\RequirePackage[latest]{latexrelease}
\documentclass[a4paper,12pt]{article}
\usepackage[utf8]{inputenc}
\usepackage[T1]{fontenc}
\usepackage[a4paper]{geometry}
\usepackage{amsmath}
\usepackage{amssymb}
\usepackage{graphicx}
\usepackage{rotating}
\usepackage{placeins}
\usepackage[slovak]{babel}
\usepackage{makeidx}
\usepackage[colorlinks=true,linkcolor=blue,urlcolor=black]{hyperref}
\usepackage{bookmark}
\usepackage{bm}
\usepackage{cleveref}
%\usepackage{tikz}
\usepackage{multicol}
\usepackage{wrapfig}
%\usepackage{booktab}
\usepackage{array}
\usepackage{siunitx}
\usepackage{lmodern}
\usepackage{ellipsis}
\usepackage{nicefrac}
%\usepackage{microtype}
%\usetikzlibrary{positioning}

\crefname{equation}{rovn.}{rovn.}
\Crefname{equation}{Rovn.}{Rovn.}
\crefname{figure}{obr.}{obr.}
\Crefname{figure}{Obr.}{Obr.}
\crefname{table}{tab.}{tab.}
\Crefname{table}{Tab.}{Tab.}
\newcommand{\crefrangeconjunction}{ - }
\renewcommand{\figurename}{Obr.}
\renewcommand{\tablename}{Tab.}
\newcommand{\overbar}[1]{\mkern 1.5mu\overline{\mkern-1.5mu#1\mkern-1.5mu}\mkern 1.5mu}
\newcommand{\overhat}[1]{\mkern 1.5mu\hat{\mkern-1.5mu#1\mkern-1.5mu}\mkern 1.5mu}
\newcommand{\abs}[1]{|#1|}
\newcommand{\lint}{\int\limits}
\newcommand{\lsum}{\sum\limits}

\usepackage{parskip}% http://ctan.org/pkg/parskip
\usepackage[dvipsnames]{xcolor}
\begin{document}
	\textbf{{\Huge Zápisnica č.5}}\\
			
	\textbf{Dátum:} 16.3.2020\\	
		
	\textbf{Účastníci:} Bc. Eva Štalmachová, Bc. Ján Urdianyk, Bc. Denis Vasko, Bc. Marek Trebuľa

	\textbf{Miesto:} Online, Microsoft Teams\\	
	
	\textbf{Čas:} 18:00 (trvanie cca 120 minút)
	
	\textbf{\textcolor{red}{Z dôvodu pandémie ochorenia COVID-19 sa toto stretunie uskutočnilo online, keďže karanténne opatrenia to neumožnili inak.}}   
    \section*{Obsah stretnutia}
    \begin{enumerate}
    	\item Do stretnutia, každý splnil svoju úlohu, ktorou bolo pripraviť príklad s postupom návrhu riadenia, simuláciou a porovnaním s PID regulátorom.
    	\item Rozdelenie príkladov bolo nasledovné Bc. Eva Štalmachová a Bc. Ján Urdianyk mali vytvoriť po jednom príklade na vstupno-stavovú spätnoväzbovú linearizáciu. Bc. Denis Vasko a Bc. Marek Trebuľa tvorili príklady na tému vstupno-výstupná spätnoväzbová linearizácia.
    	\item Na stretnutí sa vzájomne zhodnotilo vypracovanie každého člena a odporučili sa vylepšenia a prípadné korekcie.
    	\item Diskutovali sa o ďalších úlohách, ktoré je potrebné splniť.
    	\item Rozdelili sa ďalšie úlohy medzi členov tímu.
    	\item Dohodli sa ďalšom stretnutí v rámci tímu, ktoré opäť prebehne kvôli pandémii pravdepodobne online.
    \end{enumerate}    
    \section*{Úlohy}
    \begin{itemize}
    	\item Do nasledujúceho stretnutia (6.týždeň) s prof. Jánom Murgašom pripraviť a spracovať príklady určené na pedgogické účely predmetu RNS. \textcolor{OliveGreen}{Splnené - pripravené na odkonzultovanie s prof. Jánom Murgašom.}
    	\item Do pondelka (23.3.2020) sa členovia tímu pokúsia splniť nasledovné úlohy:
    	\begin{itemize}
    		\item Bc. Eva Štalmachová a Bc. Ján Urdianyk spracujú matematiku, potrebnú pre využitie týchto metód návrhu nelineárneho riadenia a vysvetlia ju na príkladoch.
    		\item Bc. Denis Vasko a Bc. Marek Trebuľa skúsia nájsť príklad vhodný na využitie týchto metód nelineárneho riadenia. Pokúsia sa vytvoriť GUI, ktoré bude slúžiť ako učebná pomôcka. Toto GUI by malo vhodne ilustrovať riešený príklad. Bc. Vasko a Bc. Trebuľa si rozdelili túto podúlohu na menšie časti. Bc. Vasko naprogramuje "backend" tj. funkcie opisujúce dynamiku, kinetiku a riadenie. Bc. Trebuľa sa zameria na "frontend" a pripraví GUI na vykreslovanie, animovanie a ovládanie dohodnutého príkladu.    		
    	\end{itemize}
    \end{itemize}

    \section*{Nasledujúce stretnutie}
    
    Stretnutie tímu online 23.3. s vypracovanými podúlohami v MS Teams.

    Naplánované na 6.týždeň (23. - 27.3.) 2020 o 10:00 v miestnosti D332 aj s prof. Jánom Murgašom. \textcolor{red}{(Toto stretnutie je zatiaľ otázne z dôvodu opatrení voči COVID - 19.)}
    
    \noindent\rule{15cm}{0.4pt}
   {\small 	\textbf{Vypracoval:} Bc. Marek Trebuľa\\
   \textbf{Dňa:} 17.3.2020 }
    

\end{document}